\chapter{Requirements Specification}\label{ch:requirements}

\section{Overview}

This chapter provides a specifications of the requirements, both functional and non-functional, that the project to be developed must satisfy, which define the overall functioning of the system to be produced. Each requirement will have a code associated for identification purposes.

This requirements specify the minimum possible constraints or characteristics that the system must fulfill. Most technology is kept outside of the requirement specification so that it does not affect the implementation or the design. To achieve a better understanding of these requirements, they are divided into the following sections, each corresponding to a different phase of product of the project.

\begin{itemize}
\item \textbf{Requirements of the ontology:} The requirements to be satisfied by the ontology as well as a minimum subset of classes and properties it must have are listed.
\item \textbf{Requirements of the server:} The requirements to be satisfied by the server and the system overall are defined in this section.
\item \textbf{Requirements of the web application:} The requirements to be satisfied by the web client are defined in this section.
\item \textbf{Requirements of the mobile application:} The requirements to be met by the mobile client are defined in this section.
\item \textbf{Non-functional requirements:} The requirements which don't specify any particular .
\end{itemize}

\section{Requirements for the ontology}

The requirement for the ontology are the ones which define the characteristics that the vocabulary to be produced must meet. This requirements are listed below:

\begin{description}
\item[RONT1] The ontology must support spatial reasoning.
\item[RONT2] The ontology must support the following inference mechanisms:
\begin{itemize}
\item disjoint classes
\item subclasses
\item subproperties
\item inverse properties
\item symmetric properties
\item functional properties
\item maximum cardinalities
\end{itemize}
\item[RONT3] The ontology must be designed following Linked Open Data best practices, so it should reuse existing vocabularies.
\item[RONT4] The ontology must represent the following types of resources:
\begin{itemize}
 \item Features
 \begin{itemize}
  \item Trails
  \item Points of Interest
  \item Geolocated Notes
 \end{itemize}
 \item Persons
 \item Multimedia Resources
 \begin{itemize}
  \item Images
  \item Video
 \end{itemize}
 \item User reviews
\end{itemize}
\item[RONT5] The resources of type \textit{Trail} must contain the information below:
\begin{itemize}
\item name
\item description
\item difficulty score
\item maximum altitude
\item minimum altitude
\item ascending slope
\item descending slope
\item posts
\item images
\item author
\item persons who traversed it
\item spatial representation
\end{itemize}
\item[RONT6] The resources of type \textit{Point of Interest} must contain the following information:
\begin{itemize}
\item name
\item description
\item altitude
\item category
\item posts
\item images
\item author
\item spatial representation
\end{itemize}
\item[RONT7] The resources of type \textit{Geolocated Note} must contain the following information:
\begin{itemize}
\item text
\item multimedia
\item author
\item privacy level
\item creation time
\item duration
\item action radius
\item spatial representation
\end{itemize}
\item[RONT8] Resources of type \textit{Person} must contain the following information:
\begin{itemize}
\item nickname
\item email
\item first name
\item family name
\item avatar
\item description
\item trail buddies
\item optionally external homepage
\item traversed trails
\item added trails
\item added points of interest
\item added notes
\item added multimedia
\end{itemize}
\item[RONT9] Multimedia resources of type \textit{Image} and \textit{Video} must contain the following information.
\begin{itemize}
\item author
\item addition date
\item download link
\item feature they belong to
\end{itemize}
\item[RONT10] Resources of type \textit{Review} must contain the following information.
\begin{itemize}
\item author
\item addition date
\item textual content
\item rating
\item related feature
\end{itemize}
\end{description}

\section{Requirements specification for the server}

The requirements for the server specify the functionality that the server must implement as well as the function and resources exposed by the API and overall inter-component communication, for the server is the central piece of the system.

\begin{description}
\item[RSV1] The system must be able to operate on Points.
\item[RSV2] The system must be able to operate on LineStrings.
\item[RSV3] The server must be able to obtain the following data from a LineString representing a trail.
 \begin{itemize}
 \item distance
 \item difficulty
 \item ascending slope
 \item descending slope
 \item maximum altitude
 \item minimum altitude
 \end{itemize}
\item[RSV4] The system must be able to create and send SPARQL queries to a data store.
\item[RSV5] The system must be able to retrieve, transform and aggregate it to the dataset, spatial data from the following sources:
  \begin{itemize}
  \item OpenStreetMap
  \item Geonames
  \end{itemize}
\item[RSV6] The server must expose a SPARQL read-only endpoint.
\item[RSV7] The server must expose a public API.
\item[RSV8] The public API must follow the REST style.
\item[RSV9] The API must expose the following resources
  \begin{itemize}
  \item Trails
  \item Points of Interest
  \item Geolocated Notes
  \item Users
  \end{itemize}
\item[RSV10] The resources must expose the following operations:
  \begin{itemize}
  \item Read
  \item Update (Except Geolocated Notes)
  \item Create
  \end{itemize}
\item[RSV11] The resources must expose information about other related resources, such as posts, images, etc.
\item[RSV12] The API must offer additional operations which expose the following functionality:
  \begin{itemize}
  \item Search
  \item Features within a area
  \item Features near each other
  \item Information about the system
  \item Information about the API
  \end{itemize} 
\item[RSV13] The server must offer secure authentication.
\item[RSV14] The system must be able to provide recommendations based on preferences and location.
\item[RSV15] The system must be able to store user preferences, for recommendation and filtering purposes.
\item[RSV16] Signing in the system will be done via e-mail and password.
\end{description}

\section{Requirements of the web application}

The requirements for the web application detail the constraints that the web based client must fulfill, as well as the functionality that need to be implemented.

\begin{description}
\item[RWEB1] The web application must work on most major browsers. The minimum browsers specified are the following:
  \begin{itemize}
  \item Google Chrome 26
  \item Mozilla Firefox 22
  \item Safari 6
  \item Internet Explorer 10
  \item Opera 12
  \end{itemize}
\item[RWEB2] The web application must communicate with the server using the public API.
\item[RWEB3] The web application will have a homepage presenting the platform and linking to the rest of the sections.
\item[RWEB4] The web application must have a section allowing the following features:
  \begin{itemize}
  \item Explore data based on its location
  \item Draw and edit trails and points of interest
  \end{itemize}
\item[RWEB5] The web application must have a section which provides the following functionality.
  \begin{itemize}
  \item Search over the data on the system
  \item Upload a trail or point of interest from a GPX file
  \item View detailed information about any data on the platform
  \end{itemize}
\item[RWEB6] The web application must have a section which provides the following features:
  \begin{itemize}
  \item Register a user on the system
  \item Log in the system
  \item Modify the profile of a user
  \end{itemize}
\item[RWEB7] All geographic data must be shown on a interactive map.
\item[RWEB8] All operations that require manipulation of spatial data have to be performed without any need of GIS knowledge from the user.
\item[RWEB9] Signing in only has to be needed for write operations.
\item[RWEB10] Registering a user via web platform will require the following information:
  \begin{itemize}
  \item User name
  \item Email
  \item Password
  \end{itemize}
\item[RWEB11] Creating a trail via web platform will require the following information:
  \begin{itemize}
  \item Name
  \item Description
  \item Geographic representation
  \end{itemize}
\item[RWEB12] Creating a point of interest via web platform will require the following information:
  \begin{itemize}
  \item Name
  \item Description
  \item Category
  \item Geographic representation
  \end{itemize}
    
\end{description}

\section{Requirements of the mobile application}

The requirements to the mobile specify the operations that can be done with the mobile application and the functionality it must implement.

\begin{description}
\item[RMOB1] The application must allow the recording of a trail.
\item[RMOB2] The application must allow exporting trails in GPX format or uploading them to the system when a network connection is available.
\item[RMOB3] The application must allow the creation of Geolocated Notes on the current position of the user.
\item[RMOB4] The application must keep track of the current position of the user at all times.
\item[RMOB5] The application must be able to receive real time notifications of nearby features.
\item[RMOB6] The application must be able to notify the user when a new feature is discovered, even if it is not currently active.
\item[RMOB7] The application must allow search and detailed view of information on the system.
\item[RMOB8] The application must provide log in and registration functionalities.
\item[RMOB9] The application can only be used by registered users.
\item[RMOB10] Registering a user via mobile platform will require the following information:
  \begin{itemize}
  \item User name
  \item Email
  \item Password
  \end{itemize}
\item[RMOB11] Creating a trail via mobile platform will require the following information:
  \begin{itemize}
  \item Name
  \item Description
  \item Recording of the trail
  \end{itemize}
\item[RMOB11] Creating a geolocated note via web mobile platform will require the following information:
  \begin{itemize}
  \item Text, Image or video
  \item Privacy level
  \item Range
  \item Coordinates
  \end{itemize} 
\item[RMOB12] Once signed in the mobile application, the system will store the password and username, for convenience purposes.
\end{description}

\section{Non-functional requirements}

The requirements not belonging to a specific category are listed in this sections. This refers to information visibility and privacy, performance and user interaction issues mainly.

\begin{description}
\item[RNF1] All the information, except a few sensible data, has to be published according to Linked Open Data best practices.
\item[RNF2] The following data must be kept private:
  \begin{itemize}
  \item User passwords
  \item The current location of the users
  \end{itemize}
\item[RNF3] The server should be able to handle high loads of connection.
\item[RNF4] The interface on the web application must be responsive, that is, the interface must adapt to mobile and tablet devices.
\item[RNF5] The web application must have a fast loading time and small memory footprint, in order to be adapted to mobile devices.
\item[RNF6] The applications must have intuitive and easy to use and read interfaces.
\item[RNF7] The style of the mobile and web applications must be similar.
\end{description}