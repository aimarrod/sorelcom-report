\chapter{Conclusions and future lines of work}\label{ch:conclusions}

\section{Overview}

This chapter compiles, as a conclusion, the opinions of the work group after the realization of the \textit{SORELCOM} project, showing their impressions on the work done and taking into account the contributions its development has made both at a personal and professional level.

\section{Goals achieved}

After following the development process, it can be stated that all the goals listed in section \ref{sec:goals} have been accomplished:

\begin{itemize}
\item A location-aware service has been created, which takes advantage of the advantages that the semantic web offers (linked-data, inferences, etc.). This service is offered to the users through a GPS community and a mobile service.

\item Several Linked Open Data vocabularies have been used on the definition of the \textit{SORELCOM} data model, thus allowing he publication of data as Linked Open Data. 

\item A model for the representation of trails and points of interest has been created. In addition, this model supports point of interest categories, compatible with OpenStreetMaps taxonomies and Geolocated Notes.

\item A method for calculating the score of a route has been created, thus giving the possibilities of evaluation. The means of corrections are provided through a trail editor interface.

\item The tools for a GPS community have been created. Whether a community is built around them, only time will tell.

\end{itemize}

\section{Considerations}

\section{Future lines of work}