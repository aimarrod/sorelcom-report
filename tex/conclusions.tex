\chapter{Conclusions and future lines of work}\label{ch:conclusions}

\section{Overview}

This chapter compiles, as a conclusion, the opinions of the work group after the realization of the \textit{SORELCOM} project, showing their impressions on the work done and taking into account the contributions its development has made both at a personal and professional level.

\section{Goals achieved}

After following the development process, it can be stated that all the goals listed in section \ref{sec:goals} have been accomplished:

\begin{itemize}
\item A location-aware service has been created, which takes advantage of the advantages that the semantic web offers (linked-data, inferences, etc.). This service is offered to the users through a GPS community and a mobile service.

\item Several Linked Open Data vocabularies have been used on the definition of the \textit{SORELCOM} data model, thus allowing he publication of data as Linked Open Data. 

\item A model for the representation of trails and points of interest has been created. In addition, this model supports point of interest categories, compatible with OpenStreetMaps taxonomies and Geolocated Notes.

\item A method for calculating the score of a route has been created, thus giving the possibilities of evaluation. The means of corrections are provided through a trail editor interface.

\item The tools for a GPS community have been created. Whether a community is built around them, only time will tell.

\end{itemize}

\section{Considerations}

The SORELCOM project has been an amazing learning experience. Through the duration of the project I have learned a lot about many things, however, the most prominent or interesting of them has been the Semantic Web. During the whole degree, the only data modeling technique we are shown is relational database modeling, thus ontology based modeling has been exciting surprise. However, the ideas and the purpose behind the Semantic Web and Linked Open Data, the idea of building a web for machines with integrated data, a better web, is what has made this concept of semantic web most interesting for me. At first I was skeptic about this technology, due to the complexities it brought to the project and the low understanding I had of it, however with time and learning I have found that I really like the ideas behind the Semantic Web and it is a field I would like to explore more in depth.  

In addition, the possibility of developing a web server using NodeJS has been something I have appreciated. I knew of this technology beforehand, however, I n ever found a opportunity to use it on a real project. Having the possibility of choosing the technology to use in the development gave me a great opportunity to finally use this tool on a decent sized project. This project has not only given me the chance of learning of The Semantic Web, I have also been able to use a technology I had been wishing to try for a while and I have learned a lot about state of the art web and mobile development tools. Not only this, but due to the involvement I've had with web development technologies and tools for the duration of the project, I have learned a lot about the web itself and about the current trends and tendencies on this field. 

Another technology that I have learned during the project is SocketIO. The technological implications may not have been so great, since during the degree we students learn to program TCP sockets, however the possibilities that this library opens are many. The ability to establish real time communication on web pages, among other environments, brings up a lot of possibilities and a lot of ideas for new project.

GeoJSON, RDF, Express, SocketIO, Phonegap, AngularJS, SASS, etc. Many technologies have been used through the projects, which I've had to learn. This has been a big time investment, however, it has also improved my learning capability and my skills as a computer engineer and programmer overall.

However, there are two sides to every coin, not every aspect of the project has been positive learning. I have worked a lot with geospatial data and learned a lot about it, however, this has served to show me that I don't really like spatial data manipulation or GIS systems overall. It has not been a big deal for the development of the project, in fact, I have to be thankful because I have been learn not only about things I find interesting but also about those that won't catch my attention.

In addition, I have found that working with a semantic backed, while giving capability of building very powerful services and providing flexibility adds a good amount of uncertainty and can increase the development time of a project. In the end, I have concluded that there is still not enough tools and libraries to develop semantic applications on some development environments, such as JavaScript. Looking at the positive side of things, it is a good opportunity to develop those libraries and provide the semantic web community with new tools to carry their job.

In general, I believe that this project has helped me grow as a person and as a professional. I have learned about state of the art technologies, the web, the semantic web, etc. I have improved my design and programming skills and organizational capabilities. In the end, I've realized how little I knew and how little I still know, but it can be assured that through this project I have improved as a computer engineer.

\section{Future lines of work}

The first iteration of the project, the version 1.0, can be considered complete. Still, there are many improvements opportunities and work to do on this platform. The following are some of the future lines of works that are being considered for the project.

\begin{itemize}
\item \textit{Extending the SORELCOM ontology:} The vocabulary developed for this project is used to represent points of interest, trails and geolocated notes. The ontology however, can be extended to represent even more features and adding additional properties to these classes.

\item \textit{Improving the difficulty calculation:} The difficulty score calculus performed is relatively simple, it takes into account only a small amount of characteristics of the trail. This formula could be extended to include more factors such as terrain type or usual climate. Improving this formula would improve the recommendation system, thus it is something to consider.

\item \textit{Improving recommendation system:} In the future, it would be possible to improve the recommendator in several ways. One of them would be going from using simple SPARQL queries to using them in conjunction with some AI system to calculate more fitting recommendations. In addition, it could be possible to add other factors to consider in the recommendation, and to apply filters as when obtaining nearby features.

\item \textit{New real-time services:} By using the platform already built as a communication nexus, it would be possible to develop new services for the mobile application. One example could be finding nearby users (as long as they allow it), which could be useful for sports buddies following the same trail.

\item \textit{Interface redesigns:} Due to time constraints, the application has been built using a styling framework and adding custom elements to it. Even if the platform is far from being a generic looking application, it still keeps some elements that make it somewhat unoriginal. In the future, it would be possible to design from the ground an interface for the SORELCOM platform, to give it a more attractive look and a distinctive appearance.

\item \textit{Improving the API:} The API currently give access to a good amount of functionalities, however it can be further extended. Taking ideas from well known GIS systems and applying them to the platform could be a technique used for the improvement.

\end{itemize}
